\documentclass[a4paper,12pt]{article}

\usepackage{latexsym}

\usepackage[top=3cm, bottom=3cm, left=2cm, right=2cm]{geometry} 

\usepackage[spanish]{babel}

\usepackage[utf8]{inputenx}

\usepackage{graphicx}

\usepackage{float}
\usepackage{fancyvrb}
\usepackage{listings}
\usepackage{color}

\definecolor{dkgreen}{rgb}{0,0.6,0}
\definecolor{gray}{rgb}{0.5,0.5,0.5}
\definecolor{orange}{rgb}{1,0.5,0.}

\lstset{frame=tb,
	language=Python,
	aboveskip=3mm,
	belowskip=3mm,
	showstringspaces=false,
	columns=flexible,
	basicstyle={\small\ttfamily},
	numbers=none,
	numberstyle=\tiny\color{gray},
	keywordstyle=\color{blue},
	commentstyle=\color{orange},
	stringstyle=\color{orange},
	breaklines=true,
	breakatwhitespace=true,
	tabsize=3
}

\def\verbatim{\tiny\@verbatim \frenchspacing\@vobeyspaces \@xverbatim}
\renewcommand\tablename{Tabla}

% Portada
\author{Martín Buchwald}

\title{75.16 Lenguajes de Programación\\
	\textbf{Trabajo Práctico: Compilador de PL/0}\\
	Facultad de Ingeniería, Universidad de Buenos Aires
	\date{Cuatrimestre II, 2015}
}

\begin{document}

\maketitle
\thispagestyle{empty}
\newpage
\tableofcontents
\newpage

\section{Objetivo}
Implementar un compilador de \texttt{PL/0} que encuentre errores y genere código ejecutable según se describe en el apunte de la cátedra.

\section{Descripción del Compilador}
El compilador fue implementado en Python\cite{python}. Para dicha implementación, se realizaron 5 módulos:
\begin{itemize}
\item Analizador Léxico: Se encarga de hacer el análisis léxico, devolviendo tokens a medida que se van pidiendo lecturas del archivo fuente. 
\item Analizador Sintáctico: Se encarga de hacer el análisis sintáctico. El tipo de compilador implementado es de tipo \textit{guiado por el parser}, por lo que será quien orqueste pidiendo los servicios del analizador sintáctico y semántico. Se basa en el grafo de sintáxis BNF\cite{bnf} (o mejor dicho, el BNF-Extendido) del lenguaje de PL/0.
\item Analizador Semántico: Se encarga de realizar el análisis semántico del archivo fuente. Almacena los identificadores en una tabla para analizar que los identificadores utilizados existan en el contexto indicado, y se utilicen de forma correcta según su tipo.
\item Generador de Código: se encarga de transformar las distintas instrucciones en código ejecutable, dependiendo del sistema operativo utilizado.
\item Compilador (llamado \texttt{comPiLad0r.py}): simplemente ejecuta el código de los demás módulos generando las conexiones entre éstos, y determinando el archivo fuente de origen, y la ruta destino.
\end{itemize}

En cada uno de los primeros 3 módulos, es importante recalcar que ante el primer error no se generará código\footnote{Se crea un generado de código nulo, que no realiza operaciones, para no tener que llenar de condicionales el código.}, y se marcarán los errores en un archivo de salida, para poder realizar las correcciones correspondientes, tratando de ser dichos mensajes lo más explicativos posible.

\subsection{Modo de ejecución}
Para poder realizar la compilación, sólo es necesario tener instalado el intérprete de Python (que viene incluido en cualquier distribución de Linux). 

Ejecutar el comando:
\lstset{language=Bash}
\begin{lstlisting}
~$ ./comPiLad0r.py path/to/sourceFile path/to/execFile
\end{lstlisting}
En caso de no indicarse ruta (y nombre) del archivo ejecutable, tomará por defecto el nombre \texttt{ejec} (guardándose en el directorio actual).

\section{Programas Correctos}
\lstset{language=Pascal}

\subsection{BIEN-00}
\subsubsection{Código Fuente}
\lstinputlisting{./Archivos/BIEN-00.PL0}
\subsubsection{Contenido Hexadecimal Completo}
\VerbatimInput{hexa/BIEN-00.txt}
\subsubsection{Listado Desensamblado de la sección \textit{.text}}
\VerbatimInput{text/BIEN-00.txt}

\subsection{BIEN-01}
\subsubsection{Código Fuente}
\lstinputlisting{./Archivos/BIEN-01.PL0}
\subsubsection{Contenido Hexadecimal Completo}
\VerbatimInput{hexa/BIEN-01.txt}
\subsubsection{Listado Desensamblado de la sección \textit{.text}}
\VerbatimInput{text/BIEN-01.txt}

\subsection{BIEN-02}
\subsubsection{Código Fuente}
\lstinputlisting{./Archivos/BIEN-02.PL0}
\subsubsection{Contenido Hexadecimal Completo}
\VerbatimInput{hexa/BIEN-02.txt}
\subsubsection{Listado Desensamblado de la sección \textit{.text}}
\VerbatimInput{text/BIEN-02.txt}


\subsection{BIEN-03}
\subsubsection{Código Fuente}
\lstinputlisting{./Archivos/BIEN-03.PL0}
\subsubsection{Contenido Hexadecimal Completo}
\VerbatimInput{hexa/BIEN-03.txt}
\subsubsection{Listado Desensamblado de la sección \textit{.text}}
\VerbatimInput{text/BIEN-03.txt}


\subsection{BIEN-04}
\subsubsection{Código Fuente}
\lstinputlisting{./Archivos/BIEN-04.PL0}
\subsubsection{Contenido Hexadecimal Completo}
\VerbatimInput{hexa/BIEN-04.txt}
\subsubsection{Listado Desensamblado de la sección \textit{.text}}
\VerbatimInput{text/BIEN-04.txt}


\subsection{BIEN-05}
\subsubsection{Código Fuente}
\lstinputlisting{./Archivos/BIEN-05.PL0}
\subsubsection{Contenido Hexadecimal Completo}
\VerbatimInput{hexa/BIEN-05.txt}
\subsubsection{Listado Desensamblado de la sección \textit{.text}}
\VerbatimInput{text/BIEN-05.txt}


\subsection{BIEN-06}
\subsubsection{Código Fuente}
\lstinputlisting{./Archivos/BIEN-06.PL0}
\subsubsection{Contenido Hexadecimal Completo}
\VerbatimInput{hexa/BIEN-06.txt}
\subsubsection{Listado Desensamblado de la sección \textit{.text}}
\VerbatimInput{text/BIEN-06.txt}


\subsection{BIEN-07}
\subsubsection{Código Fuente}
\lstinputlisting{./Archivos/BIEN-07.PL0}
\subsubsection{Contenido Hexadecimal Completo}
\VerbatimInput{hexa/BIEN-07.txt}
\subsubsection{Listado Desensamblado de la sección \textit{.text}}
\VerbatimInput{text/BIEN-07.txt}


\subsection{BIEN-08}
\subsubsection{Código Fuente}
\lstinputlisting{./Archivos/BIEN-08.PL0}
\subsubsection{Contenido Hexadecimal Completo}
\VerbatimInput{hexa/BIEN-08.txt}
\subsubsection{Listado Desensamblado de la sección \textit{.text}}
\VerbatimInput{text/BIEN-08.txt}


\subsection{BIEN-09}
\subsubsection{Código Fuente}
\lstinputlisting{./Archivos/BIEN-09.PL0}
\subsubsection{Contenido Hexadecimal Completo}
\VerbatimInput{hexa/BIEN-09.txt}
\subsubsection{Listado Desensamblado de la sección \textit{.text}}
\VerbatimInput{text/BIEN-09.txt}

\section{Programas Incorrectos}
\subsection{MAL-00}
\subsubsection{Código Fuente}
\lstinputlisting{./mal/MAL-00.PL0}
\subsubsection{Errores de Compilacion}

\begin{table}[H]
\centering
\begin{tabular}{|l|l|}
\hline
Mensajes de Error & Corrección\\
\hline
6. X = Y; 													&  X := Y;\\
7. Error Sintactico: Esperada asignacion luego de variable	& \\
\hline
11. write ('NUM=') readln (Y);								& write ('NUM=');\\ 
12. Error Sintactico: Se esperaba un END o punto y coma 		& readln (Y);\\
(;) luego de una proposicion de un Begin						&\\
\hline
14. writeln ("NUM*2=",Y*X)									& writeln ('NUM*2=',Y*X)\\
15. Error Lexico: comillas de cadena incorrecta				& \\
\hline
\end{tabular}
\caption{Errores MAL-00. Primera compilacion.}
\label{MAL-00-1}
\end{table}

\subsection{MAL-01}
\subsubsection{Código Fuente}
\lstinputlisting{./mal/MAL-01.PL0}
\subsubsection{Errores de Compilacion}

\begin{table}[H]
\centering
\begin{tabular}{|l|l|}
\hline
Mensajes de Error & Corrección\\
\hline
3. PROCEDUR POT;												& PROCEDURE POT;\\
4. Error Semantico: Identificador no encontrado (PROCEDUR)	&\\
5. Error Sintactico: Esperada asignacion luego de variable	&\\
6. Error Sintactico: Simbolo no esperado: ;Error Sintactico: &\\
Se esperaba punto (.) de finalizacion de programa			&\\
\hline
\end{tabular}
\caption{Errores MAL-01. Primera compilacion.}
\label{MAL-01-1}
\end{table}

\begin{table}[H]
\centering
\begin{tabular}{|l|l|}
\hline
Mensajes de Error & Corrección\\
\hline
6.RESU = RESU * BASE;										& RESU := RESU * BASE;\\
7.Error Sintactico: Esperada asignacion luego de variable		&\\
\hline
14.READLN(EXPO)												& READLN(EXPO);\\
15.RESU := 1;												&\\
16.Error Sintactico: Se esperaba un END o punto y coma (;) 	&\\
luego de una proposicion de un Begin							& \\
\hline
IF EXPO < 0  RESU := 0;										& IF EXPO < 0 THEN RESU := 0;\\
Error Sintactico: Se esperaba un 'then' luego de la 			&\\
condicion de un 'if'											& \\
\hline
\end{tabular}
\caption{Errores MAL-01. Segunda compilacion.}
\label{MAL-01-2}
\end{table}

\subsection{MAL-02}
\subsubsection{Código Fuente}
\lstinputlisting{./mal/MAL-02.PL0}
\subsubsection{Errores de Compilacion}

\begin{table}[H]
\centering
\begin{tabular}{|l|l|}
\hline
Mensajes de Error & Corrección\\
\hline
4. var A, B, A;												& var A, B;\\
5. Error Semantico: Identificador ya declarado				&\\
10. if Y ( 0 then B := -B;									& if Y $<$ 0 then B := -B;\\
11. Error Sintactico: Se esperaba simbolo de comparacion en 	&\\
comparacion													&\\
12. Error Sintactico: Cierre de parentesis faltante			&\\
\hline
13. while B > 0 then											& while B $>$ 0 do\\
14. Error Sintactico: Se esperaba un 'do' luego de la		&\\
condicion de un 'while'										&\\
\hline
25. readLn X;												& readLn(X);\\
26. Error Sintactico: Se esperaba un parentesis luego de		& \\
readln 														& \\
27. Error Sintactico: Se esperaba cierre de parentesis luego	& \\
de readln 													& \\
\hline
29. MULTIPLICAR;												& call MULTIPLICAR;\\
30. Error Semantico: Solo pueden utilizarse variables del 	&\\
lado izquierdo de una asignacion								&\\
31. Error Sintactico: Esperada asignacion luego de variable	&\\
writeLn ('X*Y=', Z);											&\\
Error Sintactico: Simbolo no esperado: writeLn				&\\
Error Sintactico: Se esperaba un END o punto y coma (;)		&\\
luego de una proposicion de un Begin							&\\
\hline	
\end{tabular}
\caption{Errores MAL-02. Primera compilacion.}
\label{MAL-02-1}
\end{table}

\subsection{MAL-03}
\subsubsection{Código Fuente}
\lstinputlisting{./mal/MAL-03.PL0}
\subsubsection{Errores de Compilacion}

\begin{table}[H]
\centering
\begin{tabular}{|l|l|}
\hline
Mensajes de Error & Corrección\\
\hline
1. var DO, X, Y, Q, R;										& var X, Y, Q, R;\\
2. Error Sintactico: declaracion de variable no seguida de 	&\\
un identificador												&\\
\hline
5. var V W;													& var V, W;\\
6. Error Sintactico: Se esperaba punto y coma (;) o coma (,) &\\
luego de declaracion de variable								&\\
\hline
53. end														& .\\
54. Error Sintactico: Se esperaba punto (.) de finalizacion 	&\\
de programa													&\\
\hline
\end{tabular}
\caption{Errores MAL-03. Primera compilacion.}
\label{MAL-03-1}
\end{table}

\subsection{MAL-04}
\subsubsection{Código Fuente}
\lstinputlisting{./mal/MAL-04.PL0}
\subsubsection{Errores de Compilacion}

\begin{table}[H]
\centering
\begin{tabular}{|l|l|}
\hline
Mensajes de Error & Corrección\\
\hline
1. var X, Y Z;												&var X, Y, Z;\\
2. Error Sintactico: Se esperaba punto y coma (;) o coma (,) 	&\\
luego de declaracion de variable								&\\
\hline
4. procedure MCD												& procedure MCD;\\
5. var F,G;													&\\
6. Error Sintactico: Luego de la identificacion de un 		&\\
procedimiento se esperaba por punto y coma (;)				&\\
\hline
18. do write ('X: '); readln (X);							& write('X: '); readln (X);\\
19. Error Sintactico: Se esperaba variable en asignacion, 	&\\
se encuentra la palabra reservada: do						&\\
20. Error Sintactico: Se esperaba un END o punto y coma (;) 	&\\
luego de una proposicion de un Begin							&\\
\hline
27. writeln ('MCD: ', Z); writeln ()							&writeln ('MCD: ', Z); writeln\\
28. Error Sintactico: Simbolo no esperado: ) 				&\\
\hline
\end{tabular}
\caption{Errores MAL-04. Primera compilacion.}
\label{MAL-04-1}
\end{table}

\subsection{MAL-05}
\subsubsection{Código Fuente}
\lstinputlisting{./mal/MAL-05.PL0}
\subsubsection{Errores de Compilacion}

\begin{table}[H]
\centering
\begin{tabular}{|l|l|}
\hline
Mensajes de Error & Corrección\\
\hline
CONST UNO = ;												& CONST UNO = 1;\\
Error Sintactico: asignacion de constante a un valor no 		&\\
numerico														&\\
\hline
CALL N;														& CALL INICIALIZAR;\\
Error Semantico: Solo pueden invocarse procedimientos		&\\
\hline
IF R* $<>$N THEN WRITE (R - 1, '..');						& IF R*R$<>$N THEN WRITE (R - 1, '..');\\
Error Sintactico: Simbolo no esperado: $<>$					&\\
\hline
\end{tabular}
\caption{Errores MAL-05. Primera compilacion.}
\label{MAL-05-1}
\end{table}


\subsection{MAL-06}
\subsubsection{Código Fuente}
\lstinputlisting{./mal/MAL-06.PL0}
\subsubsection{Errores de Compilacion}

\begin{table}[H]
\centering
\begin{tabular}{|l|l|}
\hline
Mensajes de Error & Corrección\\
\hline
24. DEMASIADOLARGO12345678901234567890123456789012345678901	&\\
23456789012345678901234567890,								&\\
25. Error Lexico: Identificador demasiado largo				&\\
\hline
189.  WRITELN ('******************************************	&\\
**********************************************************	&\\
**********************************************************	&\\
**********************************************************	&\\
************************************)						&************')\\
191. end.													&\\
192. Error Lexico: Cadena no finalizada antes que finalice	&\\
el archivo													&\\
193. ERROR LEXICO											&\\
194. Error Sintactico: Se esperaba un cierre de parentesis 	&\\
luego de write 												&\\
194. Error Sintactico: Se esperaba un END o punto y coma 		&\\
(;) luego de una proposicion de un Begin						&\\
195. Error Sintactico: Se esperaba punto (.) de 				&\\
finalizacion de programa										&\\
\hline
\end{tabular}
\caption{Errores MAL-06. Primera compilacion.}
\label{MAL-06-1}
\end{table}

\subsection{MAL-07}
\subsubsection{Código Fuente}
\lstinputlisting{./mal/MAL-07.PL0}
\subsubsection{Errores de Compilacion}

\begin{table}[H]
\centering
\begin{tabular}{|l|l|}
\hline
Mensajes de Error & Corrección\\
\hline
1. CONST N 20;												&CONST N = 20;\\
2. Error Sintactico: asignacion de constante esperada (=)		&\\
\hline
6. PROCEDURE TRI\#NGULO;										&PROCEDURE TRIANGULO;\\
7. VAR A, B;													&VAR A, B;\\
8. Error Sintactico: Luego de la identificacion de un 		&\\
procedimiento se esperaba por punto y coma (;)				&\\
\hline
17. WRITE ('*')												& WRITE('*');\\
18. B := B + 1												& B := B + 1\\
19. Error Sintactico: Se esperaba un END o punto y coma (;) 	&\\
luego de una proposicion de un Begin							&\\
\hline
27. A := 111111111111111111;									&A := 1;\\
28. Error Lexico: Numero demasiado largo						&\\
\hline
32. A := A + TRIANGULO										&\\
33. Error Semantico: Identificador no encontrado (TRIANGULO)	&\\
\hline
\end{tabular}
\caption{Errores MAL-07. Primera compilacion.}
\label{MAL-07-1}
\end{table}

\begin{table}[H]
\centering
\begin{tabular}{|l|l|}
\hline
Mensajes de Error & Corrección\\
\hline
28. A := A + TRIANGULO										&A := A + 1\\
29. Error Semantico: Solo pueden usarse variables o 			&\\
constantes en una expresion									&\\
\hline
\end{tabular}
\caption{Errores MAL-07. Segunda compilacion.}
\label{MAL-05-2}
\end{table}


\subsection{MAL-08}
\subsubsection{Código Fuente}
\lstinputlisting{./mal/MAL-08.PL0}
\subsubsection{Errores de Compilacion}

\begin{table}[H]
\centering
\begin{tabular}{|l|l|}
\hline
Mensajes de Error & Corrección\\
\hline
5. write (',')												& wrtite(',');\\
6. K := K + 1;												& K := K + 1;\\
7. Error Sintactico: Se esperaba un END o punto y coma (;) 	&\\
luego de una proposicion de un Begin							&\\
\hline
13. write coma;												&write(K);\\
14. Error Sintactico: Se esperaba un parentesis luego de		&\\
write 														&\\
15. Error Semantico: Solo pueden usarse variables o			&\\
constantes en una expresion									&\\
16. Error Sintactico: Se esperaba un cierre de parentesis 	&\\
luego de write 												&\\
\hline
17. call K;													& call coma;\\
18. Error Sintactico: Se esperaba un END o punto y coma (;)	&\\
luego de una proposicion de un Begin							&\\
19. Error Semantico: Solo pueden invocarse procedimientos		&\\
\hline
28. Error Sintactico: Se esperaba un END o punto y coma (;)	&end.\\
luego de una proposicion de un Begin							&\\
29. Error Sintactico: Se esperaba punto (.) de finalizacion	&\\
 de programa													&\\
\hline
\end{tabular}
\caption{Errores MAL-08. Primera compilacion.}
\label{MAL-08-1}
\end{table}

\subsection{MAL-09}
\subsubsection{Código Fuente}
\lstinputlisting{./mal/MAL-09.PL0}
\subsubsection{Errores de Compilacion}

\begin{table}[H]
\centering
\begin{tabular}{|l|l|}
\hline
Mensajes de Error & Corrección\\
\hline
2. Error Sintactico: Se esperaba punto (.) de finalizacion	& .\\
de programa													&\\
\hline
\end{tabular}
\caption{Errores MAL-09. Primera compilacion.}
\label{MAL-09-1}
\end{table}

\begin{thebibliography}{10}
\bibitem{python} \texttt{https://www.python.org/}
\bibitem{bnf} \textbf{BNF}: Backus-Naur Form. \texttt{https://en.wikipedia.org/wiki/Backus\%E2\%80\%93Naur\_Form}
\end{thebibliography}


\end{document}