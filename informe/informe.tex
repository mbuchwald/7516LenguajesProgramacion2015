\documentclass[a4paper,12pt]{article}

\usepackage{latexsym}

\usepackage[top=3cm, bottom=3cm, left=2cm, right=2cm]{geometry} 

\usepackage[spanish]{babel}

\usepackage[utf8]{inputenx}

\usepackage{graphicx}

%\usepackage{float}
\usepackage{fancyvrb}
\usepackage{listings}
\usepackage{color}

\definecolor{dkgreen}{rgb}{0,0.6,0}
\definecolor{gray}{rgb}{0.5,0.5,0.5}
\definecolor{orange}{rgb}{1,0.5,0.}

\lstset{frame=tb,
	language=Python,
	aboveskip=3mm,
	belowskip=3mm,
	showstringspaces=false,
	columns=flexible,
	basicstyle={\small\ttfamily},
	numbers=none,
	numberstyle=\tiny\color{gray},
	keywordstyle=\color{blue},
	commentstyle=\color{orange},
	stringstyle=\color{orange},
	breaklines=true,
	breakatwhitespace=true,
	tabsize=3
}

\def\verbatim{\tiny\@verbatim \frenchspacing\@vobeyspaces \@xverbatim}

% Portada
\author{Martín Buchwald}

\title{75.16 Lenguajes de Programación\\
	\textbf{Trabajo Práctico: Compilador de PL/0}\\
	Facultad de Ingeniería, Universidad de Buenos Aires
	\date{Cuatrimestre II, 2015}
}

\begin{document}

\maketitle
\thispagestyle{empty}
\newpage
\tableofcontents
\newpage

\section{Objetivo}
Implementar un compilador de \texttt{PL/0} que encuentre errores y genere código ejecutable según se describe en el apunte de la cátedra.

\section{Descripción del Compilador}
El compilador fue implementado en Python\cite{python}. Para dicha implementación, se realizaron 5 módulos:
\begin{itemize}
\item Analizador Léxico: Se encarga de hacer el análisis léxico, devolviendo tokens a medida que se van pidiendo lecturas del archivo fuente. 
\item Analizador Sintáctico: Se encarga de hacer el análisis sintáctico. El tipo de compilador implementado es de tipo \textit{guiado por el parser}, por lo que será quien orqueste pidiendo los servicios del analizador sintáctico y semántico. Se basa en el grafo de sintáxis BNF\cite{bnf} (o mejor dicho, el BNF-Extendido) del lenguaje de PL/0.
\item Analizador Semántico: Se encarga de realizar el análisis semántico del archivo fuente. Almacena los identificadores en una tabla para analizar que los identificadores utilizados existan en el contexto indicado, y se utilicen de forma correcta según su tipo.
\item Generador de Código: se encarga de transformar las distintas instrucciones en código ejecutable, dependiendo del sistema operativo utilizado.
\item Compilador (llamado \texttt{comPiLad0r.py}): simplemente ejecuta el código de los demás módulos generando las conexiones entre éstos, y determinando el archivo fuente de origen, y la ruta destino.
\end{itemize}

En cada uno de los primeros 3 módulos, es importante recalcar que ante el primer error no se generará código\footnote{Se crea un generado de código nulo, que no realiza operaciones, para no tener que llenar de condicionales el código.}, y se marcarán los errores en un archivo de salida, para poder realizar las correcciones correspondientes, tratando de ser dichos mensajes lo más explicativos posible.

\subsection{Modo de ejecución}
Para poder realizar la compilación, sólo es necesario tener instalado el intérprete de Python (que viene incluido en cualquier distribución de Linux). 

Ejecutar el comando:
\lstset{language=Bash}
\begin{lstlisting}
~$ ./comPiLad0r.py path/to/sourceFile path/to/execFile
\end{lstlisting}
En caso de no indicarse ruta (y nombre) del archivo ejecutable, tomará por defecto el nombre \texttt{ejec} (guardándose en el directorio actual).

\section{Programas Correctos}
\lstset{language=Pascal}

\subsection{BIEN-00}
\subsubsection{Código Fuente}
\lstinputlisting{./Archivos/BIEN-00.PL0}
\subsubsection{Contenido Hexadecimal Completo}
\VerbatimInput{hexa/BIEN-00.txt}
\subsubsection{Listado Desensamblado de la sección \textit{.text}}
\VerbatimInput{text/BIEN-00.txt}

\subsection{BIEN-01}
\subsubsection{Código Fuente}
\lstinputlisting{./Archivos/BIEN-01.PL0}
\subsubsection{Contenido Hexadecimal Completo}
\VerbatimInput{hexa/BIEN-01.txt}
\subsubsection{Listado Desensamblado de la sección \textit{.text}}
\VerbatimInput{text/BIEN-01.txt}

\subsection{BIEN-02}
\subsubsection{Código Fuente}
\lstinputlisting{./Archivos/BIEN-02.PL0}
\subsubsection{Contenido Hexadecimal Completo}
\VerbatimInput{hexa/BIEN-02.txt}
\subsubsection{Listado Desensamblado de la sección \textit{.text}}
\VerbatimInput{text/BIEN-02.txt}


\subsection{BIEN-03}
\subsubsection{Código Fuente}
\lstinputlisting{./Archivos/BIEN-03.PL0}
\subsubsection{Contenido Hexadecimal Completo}
\VerbatimInput{hexa/BIEN-03.txt}
\subsubsection{Listado Desensamblado de la sección \textit{.text}}
\VerbatimInput{text/BIEN-03.txt}


\subsection{BIEN-04}
\subsubsection{Código Fuente}
\lstinputlisting{./Archivos/BIEN-04.PL0}
\subsubsection{Contenido Hexadecimal Completo}
\VerbatimInput{hexa/BIEN-04.txt}
\subsubsection{Listado Desensamblado de la sección \textit{.text}}
\VerbatimInput{text/BIEN-04.txt}


\subsection{BIEN-05}
\subsubsection{Código Fuente}
\lstinputlisting{./Archivos/BIEN-05.PL0}
\subsubsection{Contenido Hexadecimal Completo}
\VerbatimInput{hexa/BIEN-05.txt}
\subsubsection{Listado Desensamblado de la sección \textit{.text}}
\VerbatimInput{text/BIEN-05.txt}


\subsection{BIEN-06}
\subsubsection{Código Fuente}
\lstinputlisting{./Archivos/BIEN-06.PL0}
\subsubsection{Contenido Hexadecimal Completo}
\VerbatimInput{hexa/BIEN-06.txt}
\subsubsection{Listado Desensamblado de la sección \textit{.text}}
\VerbatimInput{text/BIEN-06.txt}


\subsection{BIEN-07}
\subsubsection{Código Fuente}
\lstinputlisting{./Archivos/BIEN-07.PL0}
\subsubsection{Contenido Hexadecimal Completo}
\VerbatimInput{hexa/BIEN-07.txt}
\subsubsection{Listado Desensamblado de la sección \textit{.text}}
\VerbatimInput{text/BIEN-07.txt}


\subsection{BIEN-08}
\subsubsection{Código Fuente}
\lstinputlisting{./Archivos/BIEN-08.PL0}
\subsubsection{Contenido Hexadecimal Completo}
\VerbatimInput{hexa/BIEN-08.txt}
\subsubsection{Listado Desensamblado de la sección \textit{.text}}
\VerbatimInput{text/BIEN-08.txt}


\subsection{BIEN-09}
\subsubsection{Código Fuente}
\lstinputlisting{./Archivos/BIEN-09.PL0}
\subsubsection{Contenido Hexadecimal Completo}
\VerbatimInput{hexa/BIEN-09.txt}
\subsubsection{Listado Desensamblado de la sección \textit{.text}}
\VerbatimInput{text/BIEN-09.txt}

\begin{thebibliography}{10}
\bibitem{python} \texttt{https://www.python.org/}
\bibitem{bnf} \textbf{BNF}: Backus-Naur Form. \texttt{https://en.wikipedia.org/wiki/Backus\%E2\%80\%93Naur\_Form}
\end{thebibliography}


\end{document}